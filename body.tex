\section{Scope}

This Working Group is to look into the Butler layer of the Data Management System \citedsp{LDM-148,LDM-463}.
It should start immediately (August 4, 2017) and finish its remit by 29th September 2017.


\section{Responsibilities}

The Working Group (WG) has the following responsibilities:

\begin{itemize}
 \item Take input from a broad range of interested parties on the butler abstraction layer, and produce a document of Use Cases.
 \item Pin down the requirements in a written document \citedsp{LDM-556}.
 \item Rank butler requirements in terms of priority.
  \item Define mode of work in DM to achieve implementation of the Butler.
 \item The WG Chair shall convene meetings on a regular basis.
 \item The WG will make a recommendation as to whether a new Butler should be written, or whether the existing Butler code can be extended to meet the new requirements.
 \item The WG will draft a document proposing a conceptual design and implementation overview.
\end{itemize}

\section{Specific tasks}

\subsection{Draft requirements document(s)}

\citeds{LDM-556} defines all the middleware requirements, and is generated from the MagicDraw SysML model.
A section exists in \citeds{LDM-556} that will be populated with Butler requirements.
These requirements should carry priorities as used in \citeds{LSE-61}.

\subsection{Plan to go forward}

The path forward should be identified before the Working Group finishes.
This should include advice to the PM on the distribution of work among institutes (if any) and indication of individuals responsible for different components.

\subsection{Specific topics to be considered}

\begin{enumerate}
\item \textbf{Operational considerations}
The group must consider how to take care of operational constraints at NCSA.

\item \textbf{Data access}\label{da}
One major goal of Butler is to provide abstraction to data access -- that could be with multiple implementations.
How do developers access local data while allowing Condor jobs to access data on GPFS?

\item \textbf{Data format abstraction}
How do we hide from the pipelines user whether data are stored in HDF5 or FITS, or whether associated metadata is coming from a database or from a file?

\item \textbf{Data discovery}
This includes  calibration logic and mappers -- the distinction between simple data access (see \ref{da}) and associating data items with each other should be made clear.

\item \textbf{Expansion of partial data IDs} This must be well explained and pinned down for operations.

\item \textbf{Third-party instrument support}  How can we make it easier for others to use the stack
\item \textbf{Subset data staging}
\end{enumerate}

\section{Membership}

Membership on the order of seven people is optimal.
The proposed membership is:

\begin{itemize}
  \item Tim Jenness \textbf{(Chair)},
  \item John Swinbank,
  \item Maria Patterson and/or Colin Slater (UW),
  \item Jim Bosch and/or Pim Schellart (Princeton),
  \item Michelle Gower,
  \item Brian van Klaveren,
  \item SUIT representative,
  \item SQuaRE representative,
  \item Dominique Boutigny (external observer)
\end{itemize}

In addition K-T Lim is available as adviser.

\section{Reporting}

Th WG Chair shall report directly to the DM Project manager weekly.
